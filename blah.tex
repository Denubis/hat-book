\environment env_dis

\starttext

\chapter[about]{About the project}
As my digital project I will set up and test typesetting production environment for my dissertation. I will use \ConTeXt, running on Linux Ubuntu 14.04. I will also set up GitHub project to keep my work somewhere transparent, but safe.

\section[section1]{Step 1: planning, 29 April}

Here is the timeframe of my work, as discussed during FOAR705 on 29 April 2016. I have allocated 40-60 hours of my time: ideally 40 hours on set up and testing, and additional 20 hrs for potential problems that may arise.

I have set up my internal deadline as 31 May, but ideally the actual work should be done around 15 May and the extra two weeks give me some space for unexpected changes (illness, software or hardware issues, other project deadlines, personal life issues).

I have done my planning very granular, because that is how I work. I like small steps, careful planning and a lot of time (back up) for any unexpected events, such as me getting scarlet fever in the middle of the thing or breaking my vertebra when getting up in the morning (been there, done that).

Moreover, I have thought the steps ahead, so I had no problems articulating them quite clearly even with the timeframe allocated to individual elements. We will see how it goes. In general, everything takes me longer than I planned it, but I tried to be generous this time.

\startitemize[2]
\item Allocated/Spent time: 3/3 h, during the FOAR705
\stopitemize

\subsection[technical]{Technical planning}
As part of planning, I have talked to Brian about technical part of my project. We have discussed the technical options: \ConTeXt is best running on Linux, Brian suggested 14.04. As I have very old computer, I can’t do Virtual Machine or Disk Partition. I am in final stages of my PhD and I don’t want to lose programs such as  Adobe Acrobat, Photoshop, Bridge, Illustrator etc., so I cannot entirely kill my Windows OS. I have the following options:
\startitemize[packed,1]
\item Run \ConTeXt on Windows, which will likely be problematic (risk: spending too much time on it with no result, and growing frustration; my decision: tried to run it, but even the installing gave me too many errors so I decided to abandon this scenario)

\item Buy a new computer ASAP and run Virtual Machine there, or switch completely to Linux (risk: money money money. For the type of laptop I want, the price start around 1600 AUD; my decision: will try to find another option first, but in the worst case I will have to do this)

\item Borrow my friend’s old computer, try to run it Ubuntu there and see what happens (risk: spending time on it, with no or minimal result, and having to buy a new laptop after all; my decision: worth trying first)
\stopitemize

\startitemize[2]
\item Allocated/Spent time: 2/2 h (1 h talking to Brian, 1h my own research)
\stopitemize

\subsection[instructions]{Instructions for the final product}

I have spent some time looking for instruction how the final product (dissertation) should look like and as a result I have found out that I am allowed to do almost anything I want. There are some  basic rules issued by my faculty how the title page should look like, which font size to use etc., but that’s it.

\startitemize[2]
\item Allocated/Spent time: 0.5/1h (took me longer to find out, there are no definite rules! I had to manually download dissertations from our department from last 10 years to discover there are no universal rules).
\stopitemize

\section[section2]{Software preparation, 30 April}

I have decided for scenario C: borrow my friend’s old computer and install Ubuntu 14.04 on it, document to whole process in detail and then when I acquire my own new computer, recreate the environment there.

\subsection{Ubuntu 14.04 install}
The laptop I am using: Lenovo X201, Intel Core I7 processor I7, currently disk partition of Windows 7 and Ubuntu 14.04. My task is to get rid of the Windows.

\startitemize
I have watched some tutorials on Youtube how to install fresh Ubuntu:
\item \goto{Video 1}[url(https://www.youtube.com/watch?v=i_4Kh5kE3xA)]
\stopitemize
\startitemize
And how to create bootable USB stick: 
\item \goto{Video 2}[url(http://www.ubuntu.com/download/desktop/create-a-usb-stick-on-windows)]
\item \goto{Video 3}[url(https://www.youtube.com/watch?v=lIF8e_5F9B4)]
\stopitemize

I have followed the instructions:
\startitemize[n, packed]
\item Download Ubuntu and put it on the USB stick.
\item BIOS: To get to BIOS, keep pressing F2 and then F1. Go to Startup/Boot/move USB FDD as the first option, save. 
\item Then attach the USB with Ubuntu on it. 
\item Then hit save and reboot laptop.
\item Now it should all go smoothly. \footnote{Notes: OK, I was using my friend's USB ubuntu installation, which actually had Ubuntu 15.04 on it. So for a while I was running 15.04 instead of 14.04. Reinstall to 14.04 needed, ugh.
Once I have the right (14.04) booting usb ready, it all goes really fast (15 mins).}
\item Sudo apt-get update
\item Sudo apt-get upgrade
\stopitemize

\subject{Bibliography is very very very looooooooooooooooooooooooooooooong}


Aladzhov, A., Dimitrov, Y., Stamenov, S., Stamenova, V., Stoyanova, H., Ivanov, S. and Stoychev, S. 2013. {\em Arheologicheska karta na Pliska}. Sofia: Bulvest Print.

Aladzhov, D. 1974. {\em Selishta, pametnitsi, nahodki, Materiali za arheologicheska karta na Haskovski okrag}. Haskovo: Nauchno-populyarna muzeina bibliotheka.

Aladzhov, D. 1997. {\em Selishta, pametnitsi, nahodki ot Haskovskiya kray}. Haskovo: Atar-95.

Balkanski, I. 1965. {\em Arheologicheska karta na Belogradchishko}. Sofia: Otechestven front.

Balkanski, I. 1978a. {\em Kardzhali - arheologicheski pametnitsi}. Sofia: DI \quote{Septemvri}.

Balkanski, I. 1978b. {\em Krumovgrad - arheologicheski pametnitsi}. Sofia: DI \quote{Septemvri}.

Batakliev, I. 1969. {\em Pazardzhik i Pazardzhishko. Istoriko-geografski pregled}. Sofia: Profizdat.

Bobcheva, L. 1976. {\em Arheologicheska karta na Tolbuhinski okrag}. Sofia: Sofia Press.


\stoptext